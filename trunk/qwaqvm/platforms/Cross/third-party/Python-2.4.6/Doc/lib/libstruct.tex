\section{\module{struct} ---
         Interpret strings as packed binary data}
\declaremodule{builtin}{struct}

\modulesynopsis{Interpret strings as packed binary data.}

\indexii{C}{structures}
\indexiii{packing}{binary}{data}

This module performs conversions between Python values and C
structs represented as Python strings.  It uses \dfn{format strings}
(explained below) as compact descriptions of the lay-out of the C
structs and the intended conversion to/from Python values.  This can
be used in handling binary data stored in files or from network
connections, among other sources.

The module defines the following exception and functions:


\begin{excdesc}{error}
  Exception raised on various occasions; argument is a string
  describing what is wrong.
\end{excdesc}

\begin{funcdesc}{pack}{fmt, v1, v2, \textrm{\ldots}}
  Return a string containing the values
  \code{\var{v1}, \var{v2}, \textrm{\ldots}} packed according to the given
  format.  The arguments must match the values required by the format
  exactly.
\end{funcdesc}

\begin{funcdesc}{unpack}{fmt, string}
  Unpack the string (presumably packed by \code{pack(\var{fmt},
  \textrm{\ldots})}) according to the given format.  The result is a
  tuple even if it contains exactly one item.  The string must contain
  exactly the amount of data required by the format
  (\code{len(\var{string})} must equal \code{calcsize(\var{fmt})}).
\end{funcdesc}

\begin{funcdesc}{calcsize}{fmt}
  Return the size of the struct (and hence of the string)
  corresponding to the given format.
\end{funcdesc}

Format characters have the following meaning; the conversion between
C and Python values should be obvious given their types:

\begin{tableiv}{c|l|l|c}{samp}{Format}{C Type}{Python}{Notes}
  \lineiv{x}{pad byte}{no value}{}
  \lineiv{c}{\ctype{char}}{string of length 1}{}
  \lineiv{b}{\ctype{signed char}}{integer}{}
  \lineiv{B}{\ctype{unsigned char}}{integer}{}
  \lineiv{h}{\ctype{short}}{integer}{}
  \lineiv{H}{\ctype{unsigned short}}{integer}{}
  \lineiv{i}{\ctype{int}}{integer}{}
  \lineiv{I}{\ctype{unsigned int}}{long}{}
  \lineiv{l}{\ctype{long}}{integer}{}
  \lineiv{L}{\ctype{unsigned long}}{long}{}
  \lineiv{q}{\ctype{long long}}{long}{(1)}
  \lineiv{Q}{\ctype{unsigned long long}}{long}{(1)}
  \lineiv{f}{\ctype{float}}{float}{}
  \lineiv{d}{\ctype{double}}{float}{}
  \lineiv{s}{\ctype{char[]}}{string}{}
  \lineiv{p}{\ctype{char[]}}{string}{}
  \lineiv{P}{\ctype{void *}}{integer}{}
\end{tableiv}

\noindent
Notes:

\begin{description}
\item[(1)]
  The \character{q} and \character{Q} conversion codes are available in
  native mode only if the platform C compiler supports C \ctype{long long},
  or, on Windows, \ctype{__int64}.  They are always available in standard
  modes.
  \versionadded{2.2}
\end{description}


A format character may be preceded by an integral repeat count.  For
example, the format string \code{'4h'} means exactly the same as
\code{'hhhh'}.

Whitespace characters between formats are ignored; a count and its
format must not contain whitespace though.

For the \character{s} format character, the count is interpreted as the
size of the string, not a repeat count like for the other format
characters; for example, \code{'10s'} means a single 10-byte string, while
\code{'10c'} means 10 characters.  For packing, the string is
truncated or padded with null bytes as appropriate to make it fit.
For unpacking, the resulting string always has exactly the specified
number of bytes.  As a special case, \code{'0s'} means a single, empty
string (while \code{'0c'} means 0 characters).

The \character{p} format character encodes a "Pascal string", meaning
a short variable-length string stored in a fixed number of bytes.
The count is the total number of bytes stored.  The first byte stored is
the length of the string, or 255, whichever is smaller.  The bytes
of the string follow.  If the string passed in to \function{pack()} is too
long (longer than the count minus 1), only the leading count-1 bytes of the
string are stored.  If the string is shorter than count-1, it is padded
with null bytes so that exactly count bytes in all are used.  Note that
for \function{unpack()}, the \character{p} format character consumes count
bytes, but that the string returned can never contain more than 255
characters.

For the \character{I}, \character{L}, \character{q} and \character{Q}
format characters, the return value is a Python long integer.

For the \character{P} format character, the return value is a Python
integer or long integer, depending on the size needed to hold a
pointer when it has been cast to an integer type.  A \NULL{} pointer will
always be returned as the Python integer \code{0}. When packing pointer-sized
values, Python integer or long integer objects may be used.  For
example, the Alpha and Merced processors use 64-bit pointer values,
meaning a Python long integer will be used to hold the pointer; other
platforms use 32-bit pointers and will use a Python integer.

By default, C numbers are represented in the machine's native format
and byte order, and properly aligned by skipping pad bytes if
necessary (according to the rules used by the C compiler).

Alternatively, the first character of the format string can be used to
indicate the byte order, size and alignment of the packed data,
according to the following table:

\begin{tableiii}{c|l|l}{samp}{Character}{Byte order}{Size and alignment}
  \lineiii{@}{native}{native}
  \lineiii{=}{native}{standard}
  \lineiii{<}{little-endian}{standard}
  \lineiii{>}{big-endian}{standard}
  \lineiii{!}{network (= big-endian)}{standard}
\end{tableiii}

If the first character is not one of these, \character{@} is assumed.

Native byte order is big-endian or little-endian, depending on the
host system.  For example, Motorola and Sun processors are big-endian;
Intel and DEC processors are little-endian.

Native size and alignment are determined using the C compiler's
\keyword{sizeof} expression.  This is always combined with native byte
order.

Standard size and alignment are as follows: no alignment is required
for any type (so you have to use pad bytes);
\ctype{short} is 2 bytes;
\ctype{int} and \ctype{long} are 4 bytes;
\ctype{long long} (\ctype{__int64} on Windows) is 8 bytes;
\ctype{float} and \ctype{double} are 32-bit and 64-bit
IEEE floating point numbers, respectively.

Note the difference between \character{@} and \character{=}: both use
native byte order, but the size and alignment of the latter is
standardized.

The form \character{!} is available for those poor souls who claim they
can't remember whether network byte order is big-endian or
little-endian.

There is no way to indicate non-native byte order (force
byte-swapping); use the appropriate choice of \character{<} or
\character{>}.

The \character{P} format character is only available for the native
byte ordering (selected as the default or with the \character{@} byte
order character). The byte order character \character{=} chooses to
use little- or big-endian ordering based on the host system. The
struct module does not interpret this as native ordering, so the
\character{P} format is not available.

Examples (all using native byte order, size and alignment, on a
big-endian machine):

\begin{verbatim}
>>> from struct import *
>>> pack('hhl', 1, 2, 3)
'\x00\x01\x00\x02\x00\x00\x00\x03'
>>> unpack('hhl', '\x00\x01\x00\x02\x00\x00\x00\x03')
(1, 2, 3)
>>> calcsize('hhl')
8
\end{verbatim}

Hint: to align the end of a structure to the alignment requirement of
a particular type, end the format with the code for that type with a
repeat count of zero.  For example, the format \code{'llh0l'}
specifies two pad bytes at the end, assuming longs are aligned on
4-byte boundaries.  This only works when native size and alignment are
in effect; standard size and alignment does not enforce any alignment.

\begin{seealso}
  \seemodule{array}{Packed binary storage of homogeneous data.}
  \seemodule{xdrlib}{Packing and unpacking of XDR data.}
\end{seealso}
