\section{\module{tarfile} --- Read and write tar archive files}

\declaremodule{standard}{tarfile}
\modulesynopsis{Read and write tar-format archive files.}
\versionadded{2.3}

\moduleauthor{Lars Gust\"abel}{lars@gustaebel.de}
\sectionauthor{Lars Gust\"abel}{lars@gustaebel.de}

The \module{tarfile} module makes it possible to read and create tar archives.
Some facts and figures:

\begin{itemize}
\item reads and writes \module{gzip} and \module{bzip2} compressed archives.
\item creates \POSIX{} 1003.1-1990 compliant or GNU tar compatible archives.
\item reads GNU tar extensions \emph{longname}, \emph{longlink} and
      \emph{sparse}.
\item stores pathnames of unlimited length using GNU tar extensions.
\item handles directories, regular files, hardlinks, symbolic links, fifos,
      character devices and block devices and is able to acquire and
      restore file information like timestamp, access permissions and owner.
\item can handle tape devices.
\end{itemize}

\begin{funcdesc}{open}{\optional{name\optional{, mode
                       \optional{, fileobj\optional{, bufsize}}}}}
    Return a \class{TarFile} object for the pathname \var{name}.
    For detailed information on \class{TarFile} objects,
    see \citetitle{TarFile Objects} (section \ref{tarfile-objects}).

    \var{mode} has to be a string of the form \code{'filemode[:compression]'},
    it defaults to \code{'r'}. Here is a full list of mode combinations:

    \begin{tableii}{c|l}{code}{mode}{action}
    \lineii{'r'}{Open for reading with transparent compression (recommended).}
    \lineii{'r:'}{Open for reading exclusively without compression.}
    \lineii{'r:gz'}{Open for reading with gzip compression.}
    \lineii{'r:bz2'}{Open for reading with bzip2 compression.}
    \lineii{'a' or 'a:'}{Open for appending with no compression.}
    \lineii{'w' or 'w:'}{Open for uncompressed writing.}
    \lineii{'w:gz'}{Open for gzip compressed writing.}
    \lineii{'w:bz2'}{Open for bzip2 compressed writing.}
    \end{tableii}

    Note that \code{'a:gz'} or \code{'a:bz2'} is not possible.
    If \var{mode} is not suitable to open a certain (compressed) file for
    reading, \exception{ReadError} is raised. Use \var{mode} \code{'r'} to
    avoid this.  If a compression method is not supported,
    \exception{CompressionError} is raised.

    If \var{fileobj} is specified, it is used as an alternative to
    a file object opened for \var{name}.

    For special purposes, there is a second format for \var{mode}:
    \code{'filemode|[compression]'}.  \function{open()} will return a
    \class{TarFile} object that processes its data as a stream of
    blocks.  No random seeking will be done on the file. If given,
    \var{fileobj} may be any object that has a \method{read()} or
    \method{write()} method (depending on the \var{mode}).
    \var{bufsize} specifies the blocksize and defaults to \code{20 *
    512} bytes. Use this variant in combination with
    e.g. \code{sys.stdin}, a socket file object or a tape device.
    However, such a \class{TarFile} object is limited in that it does
    not allow to be accessed randomly, see ``Examples''
    (section~\ref{tar-examples}).  The currently possible modes:

    \begin{tableii}{c|l}{code}{Mode}{Action}
    \lineii{'r|'}{Open a \emph{stream} of uncompressed tar blocks for reading.}
    \lineii{'r|gz'}{Open a gzip compressed \emph{stream} for reading.}
    \lineii{'r|bz2'}{Open a bzip2 compressed \emph{stream} for reading.}
    \lineii{'w|'}{Open an uncompressed \emph{stream} for writing.}
    \lineii{'w|gz'}{Open an gzip compressed \emph{stream} for writing.}
    \lineii{'w|bz2'}{Open an bzip2 compressed \emph{stream} for writing.}
    \end{tableii}
\end{funcdesc}

\begin{classdesc*}{TarFile}
    Class for reading and writing tar archives. Do not use this
    class directly, better use \function{open()} instead.
    See ``TarFile Objects'' (section~\ref{tarfile-objects}).
\end{classdesc*}

\begin{funcdesc}{is_tarfile}{name}
    Return \constant{True} if \var{name} is a tar archive file, that
    the \module{tarfile} module can read.
\end{funcdesc}

\begin{classdesc}{TarFileCompat}{filename\optional{, mode\optional{,
                                 compression}}}
    Class for limited access to tar archives with a
    \refmodule{zipfile}-like interface. Please consult the
    documentation of the \refmodule{zipfile} module for more details.
    \var{compression} must be one of the following constants:
    \begin{datadesc}{TAR_PLAIN}
        Constant for an uncompressed tar archive.
    \end{datadesc}
    \begin{datadesc}{TAR_GZIPPED}
        Constant for a \refmodule{gzip} compressed tar archive.
    \end{datadesc}
\end{classdesc}

\begin{excdesc}{TarError}
    Base class for all \module{tarfile} exceptions.
\end{excdesc}

\begin{excdesc}{ReadError}
    Is raised when a tar archive is opened, that either cannot be handled by
    the \module{tarfile} module or is somehow invalid.
\end{excdesc}

\begin{excdesc}{CompressionError}
    Is raised when a compression method is not supported or when the data
    cannot be decoded properly.
\end{excdesc}

\begin{excdesc}{StreamError}
    Is raised for the limitations that are typical for stream-like
    \class{TarFile} objects.
\end{excdesc}

\begin{excdesc}{ExtractError}
    Is raised for \emph{non-fatal} errors when using \method{extract()}, but
    only if \member{TarFile.errorlevel}\code{ == 2}.
\end{excdesc}

\begin{seealso}
    \seemodule{zipfile}{Documentation of the \refmodule{zipfile}
    standard module.}

    \seetitle[http://www.gnu.org/software/tar/manual/html_node/tar_134.html\#SEC134]
    {GNU tar manual, Basic Tar Format}{Documentation for tar archive files,
    including GNU tar extensions.}
\end{seealso}

%-----------------
% TarFile Objects
%-----------------

\subsection{TarFile Objects \label{tarfile-objects}}

The \class{TarFile} object provides an interface to a tar archive. A tar
archive is a sequence of blocks. An archive member (a stored file) is made up
of a header block followed by data blocks. It is possible, to store a file in a
tar archive several times. Each archive member is represented by a
\class{TarInfo} object, see \citetitle{TarInfo Objects} (section
\ref{tarinfo-objects}) for details.

\begin{classdesc}{TarFile}{\optional{name
                           \optional{, mode\optional{, fileobj}}}}
    Open an \emph{(uncompressed)} tar archive \var{name}.
    \var{mode} is either \code{'r'} to read from an existing archive,
    \code{'a'} to append data to an existing file or \code{'w'} to create a new
    file overwriting an existing one. \var{mode} defaults to \code{'r'}.

    If \var{fileobj} is given, it is used for reading or writing data.
    If it can be determined, \var{mode} is overridden by \var{fileobj}'s mode.
    \begin{notice}
        \var{fileobj} is not closed, when \class{TarFile} is closed.
    \end{notice}
\end{classdesc}

\begin{methoddesc}{open}{...}
    Alternative constructor. The \function{open()} function on module level is
    actually a shortcut to this classmethod. See section~\ref{module-tarfile}
    for details.
\end{methoddesc}

\begin{methoddesc}{getmember}{name}
    Return a \class{TarInfo} object for member \var{name}. If \var{name} can
    not be found in the archive, \exception{KeyError} is raised.
    \begin{notice}
        If a member occurs more than once in the archive, its last
        occurrence is assumed to be the most up-to-date version.
    \end{notice}
\end{methoddesc}

\begin{methoddesc}{getmembers}{}
    Return the members of the archive as a list of \class{TarInfo} objects.
    The list has the same order as the members in the archive.
\end{methoddesc}

\begin{methoddesc}{getnames}{}
    Return the members as a list of their names. It has the same order as
    the list returned by \method{getmembers()}.
\end{methoddesc}

\begin{methoddesc}{list}{verbose=True}
    Print a table of contents to \code{sys.stdout}. If \var{verbose} is
    \constant{False}, only the names of the members are printed. If it is
    \constant{True}, output similar to that of \program{ls -l} is produced.
\end{methoddesc}

\begin{methoddesc}{next}{}
    Return the next member of the archive as a \class{TarInfo} object, when
    \class{TarFile} is opened for reading. Return \code{None} if there is no
    more available.
\end{methoddesc}

\begin{methoddesc}{extract}{member\optional{, path}}
    Extract a member from the archive to the current working directory,
    using its full name. Its file information is extracted as accurately as
    possible.
    \var{member} may be a filename or a \class{TarInfo} object.
    You can specify a different directory using \var{path}.
\end{methoddesc}

\begin{methoddesc}{extractfile}{member}
    Extract a member from the archive as a file object.
    \var{member} may be a filename or a \class{TarInfo} object.
    If \var{member} is a regular file, a file-like object is returned.
    If \var{member} is a link, a file-like object is constructed from the
    link's target.
    If \var{member} is none of the above, \code{None} is returned.
    \begin{notice}
        The file-like object is read-only and provides the following methods:
        \method{read()}, \method{readline()}, \method{readlines()},
        \method{seek()}, \method{tell()}.
    \end{notice}
\end{methoddesc}

\begin{methoddesc}{add}{name\optional{, arcname\optional{, recursive}}}
    Add the file \var{name} to the archive. \var{name} may be any type
    of file (directory, fifo, symbolic link, etc.).
    If given, \var{arcname} specifies an alternative name for the file in the
    archive. Directories are added recursively by default.
    This can be avoided by setting \var{recursive} to \constant{False};
    the default is \constant{True}.
\end{methoddesc}

\begin{methoddesc}{addfile}{tarinfo\optional{, fileobj}}
    Add the \class{TarInfo} object \var{tarinfo} to the archive.
    If \var{fileobj} is given, \code{\var{tarinfo}.size} bytes are read
    from it and added to the archive.  You can create \class{TarInfo} objects
    using \method{gettarinfo()}.
    \begin{notice}
    On Windows platforms, \var{fileobj} should always be opened with mode
    \code{'rb'} to avoid irritation about the file size.
    \end{notice}
\end{methoddesc}

\begin{methoddesc}{gettarinfo}{\optional{name\optional{,
                               arcname\optional{, fileobj}}}}
    Create a \class{TarInfo} object for either the file \var{name} or
    the file object \var{fileobj} (using \function{os.fstat()} on its
    file descriptor).  You can modify some of the \class{TarInfo}'s
    attributes before you add it using \method{addfile()}.  If given,
    \var{arcname} specifies an alternative name for the file in the
    archive.
\end{methoddesc}

\begin{methoddesc}{close}{}
    Close the \class{TarFile}. In write mode, two finishing zero
    blocks are appended to the archive.
\end{methoddesc}

\begin{memberdesc}{posix}
    If true, create a \POSIX{} 1003.1-1990 compliant archive. GNU
    extensions are not used, because they are not part of the \POSIX{}
    standard.  This limits the length of filenames to at most 256,
    link names to 100 characters and the maximum file size to 8
    gigabytes. A \exception{ValueError} is raised if a file exceeds
    this limit.  If false, create a GNU tar compatible archive.  It
    will not be \POSIX{} compliant, but can store files without any
    of the above restrictions. 
    \versionchanged[\var{posix} defaults to \constant{False}]{2.4}
\end{memberdesc}

\begin{memberdesc}{dereference}
    If false, add symbolic and hard links to archive. If true, add the
    content of the target files to the archive.  This has no effect on
    systems that do not support symbolic links.
\end{memberdesc}

\begin{memberdesc}{ignore_zeros}
    If false, treat an empty block as the end of the archive. If true,
    skip empty (and invalid) blocks and try to get as many members as
    possible. This is only useful for concatenated or damaged
    archives.
\end{memberdesc}

\begin{memberdesc}{debug=0}
    To be set from \code{0} (no debug messages; the default) up to
    \code{3} (all debug messages). The messages are written to
    \code{sys.stderr}.
\end{memberdesc}

\begin{memberdesc}{errorlevel}
    If \code{0} (the default), all errors are ignored when using
    \method{extract()}.  Nevertheless, they appear as error messages
    in the debug output, when debugging is enabled.  If \code{1}, all
    \emph{fatal} errors are raised as \exception{OSError} or
    \exception{IOError} exceptions.  If \code{2}, all \emph{non-fatal}
    errors are raised as \exception{TarError} exceptions as well.
\end{memberdesc}

%-----------------
% TarInfo Objects
%-----------------

\subsection{TarInfo Objects \label{tarinfo-objects}}

A \class{TarInfo} object represents one member in a
\class{TarFile}. Aside from storing all required attributes of a file
(like file type, size, time, permissions, owner etc.), it provides
some useful methods to determine its type. It does \emph{not} contain
the file's data itself.

\class{TarInfo} objects are returned by \class{TarFile}'s methods
\method{getmember()}, \method{getmembers()} and \method{gettarinfo()}.

\begin{classdesc}{TarInfo}{\optional{name}}
    Create a \class{TarInfo} object.
\end{classdesc}

\begin{methoddesc}{frombuf}{}
    Create and return a \class{TarInfo} object from a string buffer.
\end{methoddesc}

\begin{methoddesc}{tobuf}{}
    Create a string buffer from a \class{TarInfo} object.
\end{methoddesc}

A \code{TarInfo} object has the following public data attributes:

\begin{memberdesc}{name}
    Name of the archive member.
\end{memberdesc}

\begin{memberdesc}{size}
    Size in bytes.
\end{memberdesc}

\begin{memberdesc}{mtime}
    Time of last modification.
\end{memberdesc}

\begin{memberdesc}{mode}
    Permission bits.
\end{memberdesc}

\begin{memberdesc}{type}
    File type.  \var{type} is usually one of these constants:
    \constant{REGTYPE}, \constant{AREGTYPE}, \constant{LNKTYPE},
    \constant{SYMTYPE}, \constant{DIRTYPE}, \constant{FIFOTYPE},
    \constant{CONTTYPE}, \constant{CHRTYPE}, \constant{BLKTYPE},
    \constant{GNUTYPE_SPARSE}.  To determine the type of a
    \class{TarInfo} object more conveniently, use the \code{is_*()}
    methods below.
\end{memberdesc}

\begin{memberdesc}{linkname}
    Name of the target file name, which is only present in
    \class{TarInfo} objects of type \constant{LNKTYPE} and
    \constant{SYMTYPE}.
\end{memberdesc}

\begin{memberdesc}{uid}
    User ID of the user who originally stored this member.
\end{memberdesc}

\begin{memberdesc}{gid}
    Group ID of the user who originally stored this member.
\end{memberdesc}

\begin{memberdesc}{uname}
    User name.
\end{memberdesc}

\begin{memberdesc}{gname}
    Group name.
\end{memberdesc}

A \class{TarInfo} object also provides some convenient query methods:

\begin{methoddesc}{isfile}{}
    Return \constant{True} if the \class{Tarinfo} object is a regular
    file.
\end{methoddesc}

\begin{methoddesc}{isreg}{}
    Same as \method{isfile()}.
\end{methoddesc}

\begin{methoddesc}{isdir}{}
    Return \constant{True} if it is a directory.
\end{methoddesc}

\begin{methoddesc}{issym}{}
    Return \constant{True} if it is a symbolic link.
\end{methoddesc}

\begin{methoddesc}{islnk}{}
    Return \constant{True} if it is a hard link.
\end{methoddesc}

\begin{methoddesc}{ischr}{}
    Return \constant{True} if it is a character device.
\end{methoddesc}

\begin{methoddesc}{isblk}{}
    Return \constant{True} if it is a block device.
\end{methoddesc}

\begin{methoddesc}{isfifo}{}
    Return \constant{True} if it is a FIFO.
\end{methoddesc}

\begin{methoddesc}{isdev}{}
    Return \constant{True} if it is one of character device, block
    device or FIFO.
\end{methoddesc}

%------------------------
% Examples
%------------------------

\subsection{Examples \label{tar-examples}}

How to create an uncompressed tar archive from a list of filenames:
\begin{verbatim}
import tarfile
tar = tarfile.open("sample.tar", "w")
for name in ["foo", "bar", "quux"]:
    tar.add(name)
tar.close()
\end{verbatim}

How to read a gzip compressed tar archive and display some member information:
\begin{verbatim}
import tarfile
tar = tarfile.open("sample.tar.gz", "r:gz")
for tarinfo in tar:
    print tarinfo.name, "is", tarinfo.size, "bytes in size and is",
    if tarinfo.isreg():
        print "a regular file."
    elif tarinfo.isdir():
        print "a directory."
    else:
        print "something else."
tar.close()
\end{verbatim}

How to create a tar archive with faked information:
\begin{verbatim}
import tarfile
tar = tarfile.open("sample.tar.gz", "w:gz")
for name in namelist:
    tarinfo = tar.gettarinfo(name, "fakeproj-1.0/" + name)
    tarinfo.uid = 123
    tarinfo.gid = 456
    tarinfo.uname = "johndoe"
    tarinfo.gname = "fake"
    tar.addfile(tarinfo, file(name))
tar.close()
\end{verbatim}

The \emph{only} way to extract an uncompressed tar stream from
\code{sys.stdin}:
\begin{verbatim}
import sys
import tarfile
tar = tarfile.open(mode="r|", fileobj=sys.stdin)
for tarinfo in tar:
    tar.extract(tarinfo)
tar.close()
\end{verbatim}
