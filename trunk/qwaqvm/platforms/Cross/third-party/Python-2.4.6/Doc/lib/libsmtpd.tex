\section{\module{smtpd} ---
         SMTP Server}

\declaremodule{standard}{smtpd}

\moduleauthor{Barry Warsaw}{barry@zope.com}
\sectionauthor{Moshe Zadka}{moshez@moshez.org}

\modulesynopsis{Implement a flexible SMTP server}

This module offers several classes to implement SMTP servers.  One is
a generic do-nothing implementation, which can be overridden, while
the other two offer specific mail-sending strategies.


\subsection{SMTPServer Objects}

\begin{classdesc}{SMTPServer}{localaddr, remoteaddr}
Create a new \class{SMTPServer} object, which binds to local address
\var{localaddr}.  It will treat \var{remoteaddr} as an upstream SMTP
relayer.  It inherits from \class{asyncore.dispatcher}, and so will
insert itself into \refmodule{asyncore}'s event loop on instantiation.
\end{classdesc}

\begin{methoddesc}[SMTPServer]{process_message}{peer, mailfrom, rcpttos, data}
Raise \exception{NotImplementedError} exception. Override this in
subclasses to do something useful with this message. Whatever was
passed in the constructor as \var{remoteaddr} will be available as the
\member{_remoteaddr} attribute. \var{peer} is the remote host's address,
\var{mailfrom} is the envelope originator, \var{rcpttos} are the
envelope recipients and \var{data} is a string containing the contents
of the e-mail (which should be in \rfc{2822} format).
\end{methoddesc}


\subsection{DebuggingServer Objects}

\begin{classdesc}{DebuggingServer}{localaddr, remoteaddr}
Create a new debugging server.  Arguments are as per
\class{SMTPServer}.  Messages will be discarded, and printed on
stdout.
\end{classdesc}


\subsection{PureProxy Objects}

\begin{classdesc}{PureProxy}{localaddr, remoteaddr}
Create a new pure proxy server. Arguments are as per \class{SMTPServer}.
Everything will be relayed to \var{remoteaddr}.  Note that running
this has a good chance to make you into an open relay, so please be
careful.
\end{classdesc}


\subsection{MailmanProxy Objects}

\begin{classdesc}{MailmanProxy}{localaddr, remoteaddr}
Create a new pure proxy server. Arguments are as per
\class{SMTPServer}.  Everything will be relayed to \var{remoteaddr},
unless local mailman configurations knows about an address, in which
case it will be handled via mailman.  Note that running this has a
good chance to make you into an open relay, so please be careful.
\end{classdesc}
